The study of ion currents through excitable cell membranes is a key topic in the field of biophysics. The cellular membrane is a lipid bilayer containing pores known as ion channels. These ion channels allow for the controlled movement of specific ions, namely $Na^+$, $K^+$, $Cl^-$, and $Ca^{2+}$ for myocardial cells. The membrane maintains the concentration of ions inside and outside of the cell causing a concentration gradient and thus a potential difference. This potential difference is known as the transmembrane potential which drives ionic currents across a cell. These concentration gradients are maintained by voltage controlled ion pumps which move ions into and out of the cell causing what is known as an action potential. \par

The voltage changes across cardiac cell membranes that are undergoing an action potential exhibit excitable nonlinear behaviour. This behaviour is characterised by a threshold triggered fast response via positive feedback, followed by a slow response that suppresses the fast response returning the system to a resting state. The recovery phase also includes a refractory period in which the cell cannot be re-excited. It is these periodic action potentials that activate cardiac myocytes causing the contraction of the heart and subsequent heart beat.\par

Understanding the cardiac action potential and how it propagates is key to understanding and treating various cardiac conditions ranging from tachycardia to atrial fibrillation. In a time where heart disease affects 1 in 6 men and 1 in 10 women in the UK research into the field is in high demand \citep{diseaserate}. Due to the severity of the majority of cardiac conditions, in vivo studies are rare. The best way to study the cardiac action potential is by using simulation and models as tools for investigation. A combination of electronic models and computational simulations have been used to simulate various aspects of cardiac activity to different degrees of sophistication. The models we will concern ourselves with are those that simulate the dynamic activity of membrane action potentials and their propagation though cardiac tissues. The new model proposed by this paper provides an accurate simulation at a very low computational cost compared to conventional models. A method of using this lightweight model for drug treatment dosing estimates is proposed which potentially could provide patient tailored drug dosing estimates from a medical professional's personal computer.\par

Computational simulations have modelled individual cell action potentials in a variety of ways. From simple step functions  \citep{stepfunction} and rectangular functions \citep{retangularfunction} to more sophisticated models which use an imitation of the four action potential phases \citep{imitationactionpotential}. The heart has also been modelled in three dimensions via a finite element method where each element used a potential that mimicked the average over multiple cells \citep{averageovercells}. Most of these models are focused on the activity across the whole geometry, an alternative method is to focus on the individual cells and the cell-to-cell interaction on a local scale. Many of the methods which rely on detailed cellular level models are very computationally demanding. \par

The Hodgkin-Huxley model described single cell behaviour using four dimensional nonlinear differential equations \citep{hodgkinhuxley}. The Hodgkin-Huxley model was designed to model nerve action potentials and was based on the RC equivalent circuit model described in section \ref{section3}. It models each ion's current as having a linear current-voltage structure depending on the number of subunits forming the channel. Beelar \& Reuter \citep{beelerreuter} and DiFrancesco \& Nobel \citep{difrancesconobel} further developed on the ideas of the Hodgkin-Huxley model by considering 8 and 12 ionic mechanisms respectively. It is clear that, although highly sophisticated, these detailed models are highly demanding to implement computationally over larger geometries. An alternative approach to modelling action potentials was carried out by FitzHugh \citep{fitzhugh}. By taking the mathematics of Van der Pol relaxation oscillators (see appendix \ref{appendixVDP}), FitzHugh took the 4 equations of the Hodgkin-Huxley model and applied them to a 2 dimensional phase space. This became known as the FitzHugh-Nagumo (FN) model \citep{fitzhughnagumo}. A second alternative is to model the system electronically with a simple circuit that shows excitable behaviour which will be discussed further in section \ref{section3}. In both the FN and electronic models the system can be represented simply as a pair of ordinary differential equations shown below in equation \ref{eq1.1}, with excitable variable $u$ and inhibitory variable $v$. The parameter $\epsilon <<1$ is used to cause a slow response in $v$ allowing a delayed relaxation phase. This will be discussed further in section \ref{section4}.\par
\begin{equation}
    \begin{split}
    & \frac{du}{dt} = f(u,v) \\
    & \frac{dv}{dt} = \epsilon g(u,v)
    \end{split}
    \label{eq1.1}
\end{equation}
The present work aims to describe a cardiac model that accurately simulates the dynamic conduction properties of the right atrium with a focus on re-entry and re-entry related conditions. An overview of cardiac anatomy and electrophysiology will be provided in secion \ref{section2}. Both electronic models (section \ref{section3}) and computational models (section \ref{section4}) will be developed and explored with a focus on FitzHugh-Nagumo equations and equivalent circuit models based around a simple 3 transistor circuit. The tissue in question will initially be modelled as homogeneous, and will focus on combining an accurate representation of individual cell action potentials with diffusion controlled intercellular propagation (shown in section \ref{section4}). The motivation of this work is to achieve these goals and accurate simulation on a computationally lightweight model as opposed to the computational complexity of more advanced Hodgkin-Huxley and biodomain based models. These goals are achieved and discussed in full with comparison to other existing models as well as potential novel uses of the model in section \ref{section5}.