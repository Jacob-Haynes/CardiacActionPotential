The 3 transistor electronic model exhibited in section \ref{section3} created an accurate representation of a self exciting action potential, reminiscent of a SAN cell action potential as shown in figure \ref{fig3.5}. It was also shown that these individual 3 transistor `cells' can be connected together to simulate a larger system. 6 were connected in a ring and where shown to propagate the action potential around the circuit with a diffusion time between each cell, as shown in figure \ref{fig3.7}. This method of propagation is not biologically accurate but can be used to simulate the effects of tachycardia by introducing a diode and a switch in parallel between the cells. When the switch is opened, re-entry could be simulated causing arrhythmic effects as shown in figure \ref{fig3.8}. RF ablation treatment was also simulated by transforming the unidirectional block caused by the diode into an omnidirectional block by breaking the circuit with an additional switch (figure \ref{fig3.8}). \par
The electronic model, although simple, is very restrictive. The action potential shows the relevant depolarisation and repolarisation phases with a refractory period but neglects the plateau period which is so crucial for cardiac action potentials. Being a physical circuit means that large geometries become very difficult to build and measure. Individual parameters are also very hard to control due to the limit of the components available and cost. \par
The FN type equation (equation \ref{eqmodel}) derived from the 3 transistor model showed to be effective at generating a cardiac action potential for SAN cells. The simulated SAN action potential shows all the features on a typical SAN action potential as detailed in section \ref{section2} with an accurate shape. This is significant as it does so with a very simple, lightweight method and achieves a shape that matches experimental measurements as well as other more sophisticated models. Passing the output to sequential cells causes the output to merge into more of a square wave reminiscent of atrial cell potentials. The atrial output slowly evolves over 3 cells into a stable shape, the first few cells in the sequence create an intermediary output. Although the simulated atrial potential is not as accurate a shape as the simulated SAN potential, it shows all the key features detailed in section \ref{section2} at a fraction of the computational cost of more sophisticated models. \par 
This is an important result as most advanced models create action potentials that are of similar shape and, as shown in figure \ref{fig2b.5}, vary significantly with the variable experimental measurements. These advanced models are also usually based on animals rather than humans (see table \ref{ventmodletable}). Despite these issues these advanced models are still appropriate for modelling human electrophysiology. Our model achieves a similar level of accuracy in the features of the cardiac action potential as well as overall shape for both the SAN and atrial cells but at a fraction of the computational cost, particularly in comparison to the highly detailed MCZ model \citep{phdpaper}. The model has shown to accurately simulate both individual cell electrophysiology as well as propagation mechanisms and behaviours across a variable medium. \par
The SAN and first atrial cell exhibit a gradual recovery to the resting potential which is biologically accurate, however, subsequent cells lose this gradual recovery. Instead after the restitution period they immediately recover to the resting potential. This has not caused any observable unexpected effects on the conditions simulated. This restitution period is adjustable by manipulating equation \ref{eqmodel}. \par
The model shows reliable 2D wavefront propagation which interacts as expected with uni and omnidirectional blockers, however, in its current form the code does not facilitate diagonal propagation. This addition has been made in the code but in its current state is not optimised so has been emitted from simulation. The current 2D propagation model effectively simulates re-entry of the wavefront and the establishment of spiral waves in the case of the unidirectional blocker. \par
The re-entry effect spiral waves do not interact correctly with subsequent beats from the SAN in the 81 by 81 geometry. The position of the blocker, being close to the SAN, is not allowing the cells in the first few columns time to relax and thus blocking the propagating of subsequent beats. It is believed that in larger geometries this problem can be largely negated. \par
A major advantage of this computational model in comparison to many of the models discussed in literature \citep{phdpaper} is its mathematical simplicity, and thus low computational power required. Despite being so lightweight in comparison, the model has shown accurate simulation of action potentials and wavefront propagation comparable to that of the more thorough complex models. This simplicity allows the model to simulate each individual cells action potential rather than using a finite element method. \par 
Modelling each individual cell has enabled us to be able to change the properties of the equation to simulate the effect of drugs and has shown to be effective in modelling the effect of K$^+$ ion channel blockers. This model upon further refinement could be used to simulate full scale geometries, conditions, cell types, and anisotropy. This potentially could be used for individually tailored drug dosing estimates on low powered computers, by scaling to full scale geometries and using dimensional multiplicative constants to provide estimates of the extent of treatment required to halt re-entry. The current model is not optimised for multithreading and yet simulates on the scale of hours, in comparison to a typical monodomain model which runs with an average simulation time of up to 2 days on a 32 processor computer \citep{monodomain}. \par