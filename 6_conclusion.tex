Cardiac action potentials are a complicated electrophysiological system driven by ion movements across cell membranes in the heart. They are key to understanding many cardiac conditions such as re-entrant tachycardia. It is shown that Fitzhugh-Nagumo equations can be used to model cardiac action potentials via phase-plane analysis. It is also shown that Fitzhugh-Nagumo models are mathematically lightweight in comparison to more rigorous models such as bidomain models. \par
A Fitzhugh-Nagumo system can be achieved by modelling cellular ion channels using simple electronic circuits. It is shown that a 3 transistor circuit model can be used to simulate basic action potentials and re-entrant tachycardia as well as ablation treatment. The electronic model is improved upon by taking a FN equation for the 3 transistor model and modelling a matrix of cells computationally. This system allows accurate propagation of the action potential across a geometry as well as simulation of the re-entry effect and treatment via ablation. It is also shown that the parameters of the equations can be adjusted to simulate the effects of ion channel blocking drugs. \par
In comparison to other models the FN system described is incredibly lightweight and yet could easily be adapted to include some of the more advanced simulation parameters such as anisotropy. As each cell is simulated individually the model avoids relying on finite element methods and the assumptions that are made with them. The model, with refinement, could potentially be used to accurately simulate more conditions as well as personalised drug dosing and other treatments on a cellular level while still considering the large complex geometries of the system.